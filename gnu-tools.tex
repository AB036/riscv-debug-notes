\documentclass{article}

\usepackage{geometry}
 \geometry{
 a4paper,
 total={170mm,257mm},
 left=20mm,
 top=20mm,
 }

\setlength{\parskip}{7pt}
\setlength{\parindent}{0pt}

\title{GNU Toolchain}
\date{2019-09-27}

\usepackage{hyperref}
\usepackage{listings}
\usepackage{amssymb}
\usepackage{graphicx}
\graphicspath{ {./images/} }
\usepackage{float}

\lstset{
	breaklines=true,
	frame=single
}

\begin{document}
	\maketitle
	
	\section{About}
	
	The GNU toolchain is a collection of programming tools. A version for RISC-V is available at \url{https://github.com/riscv/riscv-gnu-toolchain}.
	
	The objective of this document is to present the relevant tools for compiling and debugging software, and to offer quick guides and examples about using them.
	
	\newpage
	\section{Platforms}
	
	\subsection{Target Triplet}
	
	A target triplet is used to identify systems. It allows build system to know if the code will run.
	
	Some examples are :
	
	\vspace{-\topsep}
	\begin{itemize}
	\item x86\_64-linux-gnu
	\item x86\_64-unknown-freebsd
	\item i686-pc-linux-gnu
	\item arm-linux-gnueabihf
	\item riscv32-unknown-elf
	\end{itemize}
	
	Target triplets have the following structure: \textbf{machine} - \textbf{vendor} - \textbf{operating system}
	
	The \textbf{vendor} field may be left out. In that case, it is assumed to be the default (unknown). As such, x86\_64-unknown-freebsd and x86\_64-freebsd are the same.
	
	The \textbf{operating system} field can be composed of two words. For example, x86\_64-linux-gnu is actually:
	
	\vspace{-\topsep}
	\begin{itemize}
	\item machine: x86\_64
	\item vendor: unknown (left out)
	\item os: linux-gnu
	\end{itemize}
	
	
	\subsection{Build, Host, Target}
	
	The \textbf{build platform} is the one on which the compilation tools are executed.
	
	The \textbf{host platform} is the one on which the code will eventually run.
	
	The \textbf{target platform} of a compiler is the one the compiler generates code for.
	
	The GNU tools state the platforms they are configured for. The target triplet can be included in the filename of the binaries:
	
	\vspace{-\topsep}
	\begin{itemize}
	\item aarch64-linux-gnu-gcc
	\item riscv32-unknown-elf-gdb
	\item riscv64-unknown-elf-objdump
	\end{itemize}
	
	You may also find it with the -v option:
	
	\begin{lstlisting}[language=bash]
    $ gcc -v
    [...]
    Target: x86_64-linux-gnu
    Configured with: [...] --build=x86_64-linux-gnu --host=x86_64-linux-gnu --target=x86_64-linux-gnu
    [...] 
    \end{lstlisting}
    
    Or the -dumpmachine option:
    
    \begin{lstlisting}[language=bash]
    $ gcc -dumpmachine
    x86_64-linux-gnu
    \end{lstlisting}
    
    \subsection{RISC-V extensions}
    
    The RISC-V instruction set architecture can use extensions. There is the core ISA, standard extensions, and custom extensions.
    
    Each extension brings its own instructions. Consequently, the toolchain must be configured with the correct architecture, or you might encounter errors. This is especially true with custom non-standard extensions.
    
    You can find information about the target architecture:
    
    \begin{lstlisting}[language=bash]
    $ riscv32-unknown-elf-gcc -v
    [...]
    Target: riscv32-unknown-elf
    Configured with: [...] --with-arch=rv32i [...]
    [...] 
    \end{lstlisting}
    
    \newpage
    \section{GCC}
    
    GCC is the GNU Compiler Collection. It is a compiler system which supports many languages, including C and C++.
    
    \subsection{Basic usage}
    
    If \textit{hello.c} is a C source code file, you can compile it into an executable file \textit{hello.exe} with
    
    \begin{lstlisting}[language=bash]
    $ gcc -o hello.exe hello.c
    \end{lstlisting}
    
    \subsection{Options}
    
    GCC has \textbf{many} options. A full list is available at \url{https://gcc.gnu.org/onlinedocs/gcc/Option-Summary.html}
    
	\subsubsection{Debugging}    
    
    Regarding debugging, important options are:
    
    \vspace{-\topsep}
	\begin{itemize}
	\item \textbf{-g} : produce debugging information that GDB can use
	\item \textbf{-Og} : optimization level of choice for debugging
	\end{itemize}
	
	
    \subsubsection{Linker}
    
    To produce executable files, GCC includes a linker. You may need to provide your own linker script for some target platforms, such as bare-board targets without an operating system.
    
    \vspace{-\topsep}
	\begin{itemize}
	\item \textbf{-T script} : use \textit{script} as the linker script
	\end{itemize}
    
    \newpage
    \section{GDB}
    
    GDB is the GNU debugger. It can be used to:
    
    \vspace{-\topsep}
	\begin{itemize}
	\item load a program to a remote system
	\item read the target's registers
	\item step through a program
	\item debug a program
	\end{itemize}
	
	A full guide of GDB is available at \url{https://sourceware.org/gdb/download/onlinedocs/gdb/index.html}
	
	\subsection{Using GDB for a remote RISC-V target}
	
	This section presents a way of debugging a program for a bare-metal application running on a RISC-V target. Assuming you have a \textit{hello.c} file and compiled it into a \textit{hello.exe} executable file:
	
	\vspace{-\topsep}
	\begin{itemize}
	\item launch gdb with the program
	
	\begin{lstlisting}[language=bash]
    $ riscv32-unknown-elf-gdb hello.exe
    \end{lstlisting}
    
    \item Connect to a GDB translator such as OpenOCD
    
    \begin{lstlisting}[language=bash]
    (gdb) target remote [host]:[port]
    \end{lstlisting}
    
    \item load the program
    
    \begin{lstlisting}[language=bash]
    (gdb) load
    \end{lstlisting}
	\end{itemize}
	
	\subsection{Breakpoints, Continuing, Stepping}
    
    Setting a breakpoint:
    \begin{lstlisting}[language=bash]
    (gdb) b location
    \end{lstlisting}
    
    \textit{location} can be a function name, a line number, or an address of an instruction
    
    \begin{lstlisting}[language=bash]
    (gdb) b main
    (gdb) b 20
    (gdb) b *0x12345678
    \end{lstlisting}
    
    Resume program execution, i.e. continuing:
    \begin{lstlisting}[language=bash]
    (gdb) c
    \end{lstlisting}
    
    Step through a single source line:
    \begin{lstlisting}[language=bash]
    (gdb) s
    \end{lstlisting}
    
    Step through a single instruction:
    \begin{lstlisting}[language=bash]
    (gdb) si
    \end{lstlisting}
    
    \subsection{Layouts}
    
    Layouts transform the GDB interface and provide more information about the program.
    
    \begin{lstlisting}[language=bash]
    (gdb) layout layout_name
    \end{lstlisting}
    
    Layout names are:
    
    \vspace{-\topsep}
	\begin{itemize}
	\item src : displays source
	\item asm : displays disassembly
	\item split : displays source and disassembly
	\item regs : displays registers
	\end{itemize}
	
	\subsection{Other}
	
	Show all current breakpoints:
	\begin{lstlisting}[language=bash]
    (gdb) info breakpoints
    \end{lstlisting}
    
    Show all registers and their contents:
    \begin{lstlisting}[language=bash]
    (gdb) info all-registers
    \end{lstlisting}
    
    Show an expression's value:
    \begin{lstlisting}[language=bash]
    (gdb) print expression
    \end{lstlisting}
    
    Jump to a \textit{location}:
    \begin{lstlisting}[language=bash]
    (gdb) j location
    \end{lstlisting}
    
    \newpage
    \section{objdump}
    
    objdump is a program that can be used to disassemble an executable file.
    
    The program has many options that you can check here: \url{https://sourceware.org/binutils/docs/binutils/objdump.html}
    
    \subsection{Example}
    
    The following example uses the options \textbf{-D} (disassemble the contents of all sections) and \textbf{-S} (display source code intermixed with disassembly).
    
    \begin{lstlisting}[language=bash]
    $ riscv32-unknown-elf-objdump -D -S hello.exe
    \end{lstlisting}
    
    This command prints the assembly to stdout. You can pipe it to save it in a file.
    
    \begin{lstlisting}[language=bash]
    $ riscv32-unknown-elf-objdump -D -S hello.exe > hello.D
    \end{lstlisting}
    
	
\end{document}